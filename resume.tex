% % (c) 2002 Matthew Boedicker <mboedick@mboedick.org> (original author) http://mboedick.org
% % (c) 2003-2007 David J. Grant <davidgrant-at-gmail.com> http://www.davidgrant.ca
% % (c) 2008 Nathaniel Johnston <nathaniel@nathanieljohnston.com> http://www.nathanieljohnston.com
% %
% % (c) 2012 Scott Clark <sc932@cornell.edu> cam.cornell.edu/~sc932
% %
% %This work is licensed under the Creative Commons Attribution-Noncommercial-Share Alike 2.5 License. To view a copy of this license, visit http://creativecommons.org/licenses/by-nc-sa/2.5/ or send a letter to Creative Commons, 543 Howard Street, 5th Floor, San Francisco, California, 94105, USA.





\documentclass{article}

\usepackage{verbatim}
\usepackage{hyperref}
\hypersetup{
  colorlinks=true,
  urlcolor=black,      % color of file links
  }
%\usepackage[scaled]{helvet}
%\renewcommand*\familydefault{\sfdefault}

\newenvironment{longversion}{}{} % use this to show longversion
%\newenvironment{longversion}{\comment}{\endcomment} % use this to hide  longversion
\newenvironment{shortversion}{}{} % use this to show shortversion
%\newenvironment{shortversion}{\comment}{\endcomment} % use this to hide shortversion

\usepackage{changepage}
\usepackage{tabularx}
\usepackage{setspace}
\usepackage{url}
\usepackage{sectsty}
\usepackage[letterpaper,margin=0.55in]{geometry}
\pagestyle{empty}

\newenvironment{absolutelynopagebreak}
  {\par\nobreak\vfil\penalty0\vfilneg
   \vtop\bgroup}
  {\par\xdef\tpd{\the\prevdepth}\egroup
   \prevdepth=\tpd}

% indentsection style, used for sections that aren't already in lists
% that need indentation to the level of all text in the document
\newenvironment{indentsection}[1]%
{\begin{list}{}%
	{\setlength{\leftmargin}{#1}}%
	\item[]%
}
{\end{list}}

% opposite of above; bump a section back toward the left margin
\newenvironment{unindentsection}[1]%
{\begin{list}{}%
	{\setlength{\leftmargin}{-0.5#1}}%
	\item[]%
}
{\end{list}}

% format two pieces of text, one left aligned and one right aligned
\newcommand{\headerrow}[2]
{\begin{tabular*}{\linewidth}{l@{\extracolsep{\fill}}r}
	#1 &
	#2 \\
\end{tabular*}}

% make "C++" look pretty when used in text by touching up the plus signs
\newcommand{\CPP}
{C\nolinebreak[4]\hspace{-.05em}\raisebox{.22ex}{\footnotesize\bf ++}}

%edit the section font and style
\sectionfont{\normalfont\sectionrule{0pt}{0pt}{-4pt}{1pt}}

%make all sections cap and first letter capital
\newcommand{\tmpsection}[1]{}
\let\tmpsection=\section
\renewcommand{\section}[1]{\tmpsection*{\textsc{#1}}}

%set the line spacing
\setstretch{1.10}


\begin{document}
%TITLE
%todo change the email address and website address to your new address.

\begin{center}
 {\Large \textsc{Suyash Agrawal} }\\ 
\begin{tabular}{ l p{8cm} r }
    & &   \\
  \href{http://www.cse.iitd.ernet.in/}{Computer Science and Engineering} & & suyash1212@gmail.com \\
  Indian Institute of Technology, Delhi
%    & & +91-9717060183 \\
  & & \href{https://github.com/ozym4nd145/}{github.com/ozym4nd145} \\
  & & \href{https://www.linkedin.com/in/suyash1012}{linkedin.com/in/suyash1012}\\
\end{tabular}
\end{center}


\section{Academic Details}

\begin{center}
\begin{tabular}{ |c | c | c | c |}
\hline
Year & Degree & Institute & CGPA/Percentage \\ 
\hline
2015-2019 & B.Tech in Computer Science & Indian Institute of Technology & 9.898 \\ 
(Current) & and Engineering & Delhi & \textbf{Institute Rank 1}\\
\hline


2015 & Class XII, CBSE & Vishva Bharti Public School & 96.4\% \\ 

\hline
2013 & Class X, CBSE & Christ Jyoti Senior Secondary School & 10.00 \\  \hline
\end{tabular}
\end{center}

\section{Scholastic Achievements}
\begin{itemize}
    \setlength\itemsep{0.0em}
    \item Secured \textbf{All India Rank 69} in Joint Entrance Exam Advanced - 2015 among 150 Thousand candidates.
    \item \textbf{Institute Rank 1}. Consistently maintaining institute rank in top 3 among 850 students during academic years 2015-2017. IIT Delhi granted scholarship for the same.
    \item Selected for \textbf{ACM-ICPC 2017 and 2018} Regional Round with teams competing from all over India.
    \item Runner up in \textbf{Microsoft's Code.Fun.Do} campus wide Hackathon in 2016, 2017 and 2018.
    % \item Ranked in \textbf{Top 0.01\%} among 1.4 million candidates appearing in Joint Entrance Examination(JEE Mains-2015).
    \item Selected as \textbf{KVPY} Scholar in `Kishore Vaigyanik Protsahan Yojana' by Indian Institute of Science given to top 1\%.
    \item Became a National Talent Search Examination (\textbf{NTSE}) scholar for being in top 1000 at National level in 2013.
\end{itemize}

\section{Major Projects}
% \begin{spacing}{1}
\begin{list} {\labelitemi}{\leftmargin=0em}
\setlength{\leftmargin}{0pt}

    \item[]
    \headerrow {\textbf{Automated Video Description}}{Prof. Subhashis Banerjee, May-July 2017}
    \begin{itemize}
    \setlength\itemsep{0.0em}
        \item Built software for generating novel description of short video clips.
        \item Used transfer learning in encoder by employing state of art CNN (Inception V4) to encode individual video frames.
        \item Designed encoder decoder network architecture consisting of Multilayered LSTMs to achieve this translation.
        \item Experimented with Data Augmentation, Audio Features, Attention models, Loss metrics to improve performance.
        \item Explored its application in areas like video surveillance and in helping visually impaired.
    \end{itemize}

    \item[]
    \headerrow {\textbf{Automated Image Captioning}}{Prof. Subhashis Banerjee, Jan-April 2017}
    \begin{itemize}
    \setlength\itemsep{0.0em}
        \item Developed a software to automatically generate captions for images.
        \item Used an encoder decoder network similar to that in machine translation for generating captions.
        \item Used Inception V4 network to generate image embedding using transfer learning.
        \item Used Multilayered LSTM network to decode image embedding into natural language sentence.
        \item Achieved baseline performance of paper Show and Tell by Vinyals et al.
    \end{itemize}

    \item[]
    \headerrow {\textbf{Pipelined MIPS Simulator with debugger and cache simulator}}{Prof. Kolin Paul, Mar-April 2017}
    \begin{itemize}
    \setlength\itemsep{0.0em}
        \item Developed a pipelined MIPS simulator in C that supported animation of instructions being executed in different stages.
        \item Simulated all pipelined stages in parallel using threads (pthreads).
        \item Designed a trace based cache simulator and debugger for the processor with various configuration options.
        \item Used SVG to show current instruction in each stage and Javascript, CSS for styling.
    \end{itemize}

\end{list}
% \end{spacing}

\begin{absolutelynopagebreak}
\begin{longversion}
\section{Other Projects}
\begin{list} {\labelitemi}{\leftmargin=0em}
\setlength{\leftmargin}{0pt}
% \setlength\itemsep{5em}
%\begin{itemize}

\item[]
\headerrow {\textbf{Flashsubs}} {Independent Project, Mar - April 2016}
\begin{itemize} \item[]
Created a software for management of media library (Movies/TV Series) which intelligently fetches subtitles and renames obfuscated media files. It uses file hash to identify media in online database and uses filename for identification as last resort. It automatically downloads metadata like IMDb rating, plot, actors list etc. The link to project is \href{https://github.com/ozym4nd145/FlashSubs}{here}.
\end{itemize}

\headerrow {\textbf{\href{http://cap.rdv-iitd.com}{Rendezvous CAP Portal}}} {Independent Project, June - July 2017}
\begin{itemize} \item[]
Designed and developed the entire backend of Campus representative portal in NodeJS. Used Amazon DynamoDB for database and designed ingenious table structure to minimize the number of tables required. Implemented token based authentication along with OAuth Login and designed RESTful APIs for the entire competitive portal.
\end{itemize}


\item[]
\headerrow {\textbf{Multicycle ARM Processor}}{Prof. Anshul Kumar, Mar - April 2017}
\begin{itemize} \item[]
Developed a Multicycle ARM processor in VHDL on FPGA. Modelled memory as slave and used AHB Lite bus for connection. Implemented UART for memory and extended processor with 7 segment display and interrupt controller.
\end{itemize}

\item[]
\headerrow {\textbf{Prolog Interpreter}}{Prof. Sanjiva Prasad, Mar - April 2017}
\begin{itemize} \item[] 
Designed a prolog interpreter written in OCaml. It used OCaml-lex for token generation and OCaml-yacc for parsing. Backtracking and rule unification were used to implement the relational backbone of the interpreter.
\end{itemize}

\item[]
\headerrow{ \textbf{Krivine and SECD Machine}} {Prof. Sanjiva Prasad, April - May 2017}
\begin{itemize} \item[] 
Implemented a compiler with krivine and secd machine in Ocaml. A Lex Scanner converted program to tokens which were converted to an Abstract Syntax Tree using Recursive Descent Parser. The AST was type checked and a low level code was generated, which was executed by the machines. Machines also supported features like scoping,recursion etc.
\end{itemize}

% \item[]
% \headerrow { \textbf{AC Circuit Solver}} {Prof. Kolin Paul, Feb - Mar 2017}
% \begin{itemize} \item[]
% Designed and developed a AC Circuit solver in C which generates an SVG of input circuit and gives steady state solution of circuit. It used Lex and Yacc to parse netlist and Gaussian Elimination to solve node equations of circuit. 
% \end{itemize}


\end{list}
%\end{itemize}

% \pagebreak

% \textit{*All projects are available in my github profile}
\end{longversion}

\begin{longversion}
\section{Relevant Courses}
\begin{itemize}
\setlength\itemsep{-1em}
\item \textbf{Computer Science:} \hfill \textit{(*Courses currently pursuing)}\\
Computer Vision*, Algorithm Design*, AI*, Networks*, Logic for CS*, Cryptography*, Programming Languages, Computer Architecture, Design Practices, Data Structures \& Algorithms, Discrete Mathematics, Digital Logic\\

\item \textbf{Mathematics and Electrical:}
Signals \& Systems, Prob. \& Stochastic Processes, Calculus, Linear Algebra.\\

\item \textbf{Online:}
Deep Learning (\href{http://www.fast.ai/}{Fast.ai}), Intro to Machine Learning (Stanford, Coursera), Intro to CS (CS50, Harvard).
\end{itemize}
% \textit{*Courses currently pursuing}
\end{longversion}


\begin{longversion}
\section{Technical Skills}\begin{itemize}
\item \textbf{Programming Languages:}  C, \CPP, Python, Java, JavaScript, NodeJS, OCaml, VHDL, C\#, Matlab.
\item \textbf{Frameworks:} ExpressJS, Loopback, Django, Web2Py, Bootstrap, JQuery, MongoDB, DynamoDB, Tensorflow

% \item \textbf{Programming Environments:} Git, Android Studio, LaTeX, Visual Studio, Xilinx ISE Design Suite

\end{itemize}

\end{longversion}

\section{Extra Curricular Activities}
%\setlength{\leftmargin}{0pt}


\begin{itemize}
    \setlength\itemsep{0em}
    \item Co-founded \textbf{Development Club} in IIT Delhi to spread software development culture in college.
    \item Coordinator at \textbf{Coding Club}, responsible for organizing all competitive coding related events at IIT Delhi.
    \item System Administrator in \textbf{Updaters Group}, handling all the system administration work of CSE Dept. IIT Delhi.
    \item Elected as \textbf{Microsoft Student Partner}, responsible for representing Microsoft at IIT Delhi. 
    \item \textbf{Technical Coordinator} at \href{http://tryst-iitd.com/}{Tryst 2018}, responsible for back-end of the techincal festival.
    % \item \textbf{Technical Activity Head} at \href{http://rdv-iitd.com/}{Rendezvous 2017}, responsible for all the technical back-end of the cultural festival.
    % \item Appointed as 3rd Year Representative of Kumaon Hostel at IIT Delhi.
    % \item Developed a chat bot named CampusBot during Microsoft Code.Fun.Do to fulfill the basic needs of college students.
    % \item Won MockStock , an online trading competition , in \href{http://tryst-iitd.com/iitd}{Tryst 2017}.
     
\end{itemize}
\end{absolutelynopagebreak}


\end{document}
