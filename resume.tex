% % (c) 2002 Matthew noedicker <mboedick@mboedick.org> (original author) http://mboedick.org

% % (c) 2003-2007 David J. Grant <davidgrant-at-gmail.com> http://www.davidgrant.ca
% % (c) 2008 Nathaniel Johnston <nathaniel@nathanieljohnston.com> http://www.nathanieljohnston.com
% %
% % (c) 2012 Scott Clark <sc932@cornell.edu> cam.cornell.edu/~sc932
% %
% %This work is licensed under the Creative Commons Attribution-Noncommercial-Share Alike 2.5 License. To view a copy of this license, visit http://creativecommons.org/licenses/by-nc-sa/2.5/ or send a letter to Creative Commons, 543 Howard Street, 5th Floor, San Francisco, California, 94105, USA.





\documentclass{article}

\usepackage{verbatim}
\usepackage{hyperref}
\hypersetup{
  colorlinks=true,
  urlcolor=blue,      % color of file links
  }
%\usepackage[scaled]{helvet}
%\renewcommand*\familydefault{\sfdefault}

\newenvironment{longversion}{}{} % use this to show longversion
%\newenvironment{longversion}{\comment}{\endcomment} % use this to hide  longversion
\newenvironment{shortversion}{}{} % use this to show shortversion
%\newenvironment{shortversion}{\comment}{\endcomment} % use this to hide shortversion

\usepackage{changepage}
\usepackage{tabularx}
\usepackage{setspace}
\usepackage{url}
\usepackage{sectsty}
\usepackage[letterpaper,margin=0.50in]{geometry}
\pagestyle{empty}

\newenvironment{absolutelynopagebreak}
  {\par\nobreak\vfil\penalty0\vfilneg
   \vtop\bgroup}
  {\par\xdef\tpd{\the\prevdepth}\egroup
   \prevdepth=\tpd}

% indentsection style, used for sections that aren't already in lists
% that need indentation to the level of all text in the document
\newenvironment{indentsection}[1]%
{\begin{list}{}%
	{\setlength{\leftmargin}{#1}}%
	\item[]%
}
{\end{list}}

% opposite of above; bump a section back toward the left margin
\newenvironment{unindentsection}[1]%
{\begin{list}{}%
	{\setlength{\leftmargin}{-0.5#1}}%
	\item[]%
}
{\end{list}}

% format two pieces of text, one left aligned and one right aligned
\newcommand{\headerrow}[2]
{\begin{tabular*}{\linewidth}{l@{\extracolsep{\fill}}r}
	#1 &
	#2 \\
\end{tabular*}}

% make "C++" look pretty when used in text by touching up the plus signs
\newcommand{\CPP}
{C\nolinebreak[4]\hspace{-.05em}\raisebox{.22ex}{\footnotesize\bf ++}}

%edit the section font and style
\sectionfont{\normalfont\sectionrule{0pt}{0pt}{-4pt}{1pt}}

%make all sections cap and first letter capital
\newcommand{\tmpsection}[1]{}
\let\tmpsection=\section
\renewcommand{\section}[1]{\tmpsection*{\textsc{#1}}}

%set the line spacing
\setstretch{1.10}


\begin{document}
%TITLE
%todo change the email address and website address to your new address.

\begin{center}
 {\Large \textsc{Suyash Agrawal} }\\ 
\begin{tabular}{ l p{8cm} r }
    & &   \\
  \href{http://www.cse.iitd.ernet.in/}{Computer Science and Engineering} & & suyash1212@gmail.com \\
  Indian Institute of Technology, Delhi
%    & & +91-9717060183 \\
  & & \href{https://github.com/ozym4nd145/}{github.com/ozym4nd145} \\
  & & \href{https://www.linkedin.com/in/suyash1012}{linkedin.com/in/suyash1012}\\
\end{tabular}
\end{center}


\section{Academic Details}

\begin{center}
\begin{tabular}{ |c | c | c | c |}
\hline
Year & Degree & Institute & CGPA/Percentage \\ 
\hline
2015-2019 & B.Tech in Computer Science & Indian Institute of Technology & 9.885 \\ 
(Current) & and Engineering & Delhi & \textbf{Institute Rank 3}\\
\hline


2015 & Class XII, CBSE & Vishva Bharti Public School & 96.4\% \\ 

\hline
2013 & Class X, CBSE & Christ Jyoti Senior Secondary School & 10.00 \\  \hline
\end{tabular}
\end{center}

\section{Scholastic Achievements}
\begin{itemize}
    \setlength\itemsep{0.0em}
    \item Secured \textbf{All India Rank 69} in Joint Entrance Exam Advanced - 2015 among 150,000 candidates.
    \item \textbf{Institute Rank 3}. Consistently ranked in top 3 in the institute during academic years 2015-2018.
    \item Qualified for \textbf{ACM-ICPC 2017 and 2018} Regional Round with teams competing from all over India.
    \item Among \textbf{top 3} teams in \textbf{Microsoft's Code.Fun.Do} campus wide Hackathon in \textbf{2016, 2017 and 2018}.
    % \item Ranked in \textbf{Top 0.01\%} among 1.4 million candidates appearing in Joint Entrance Examination(JEE Mains-2015).
    \item Became a National Talent Search Examination (\textbf{NTSE}) scholar for being in \textbf{top 1000} at National level in 2013.
    \item Selected as \textbf{KVPY} Scholar in `Kishore Vaigyanik Protsahan Yojana' by Indian Institute of Science given to \textbf{top 1\%}.
\end{itemize}

\section{Internships \& Major Projects}
% \begin{spacing}{1}
\begin{list} {\labelitemi}{\leftmargin=0em}
\setlength{\leftmargin}{0pt}

    \item[]
    \headerrow {\textbf{Order Entry Service}}{\underline{Risk Technology, Tower Research Capital}}
    \headerrow {\emph{Summer Internship}}{\emph{May-July 2018}}
    \begin{itemize}
    \setlength\itemsep{0.0em}
	\item Designed a modern Web based architecture for Order Entry service used for sending manual orders to market.
	\item Used C++ to create a micro-service for managing raw socket connections with Trade servers and sending market orders. 
	\item Created Order Manager service in Golang to handle multiple users and serve as a broker between user and markets.
	\item Developed an extensible API using GRPC and Protobufs for communication between Order Manager and multiple Pods.
	\item Devised various mechanisms to extensively handle connection drops and corresponding updation in the whole network. Tested the whole network architecture extensively for fault tolerance in various connections.
    \item Used REST APIs to standardize order sending protocol and WebSockets for realtime notification of orders on Dashboard.
	% \item Created a highly available micro-service in C++ to handle low latency market orders.
	% \item Developed a robust micro-service using Golang for order management and routing.
	% \item Used GRPC and Protobufs for network communication between micro-services.
	% \item Created Dynamic UI using Javascript and employed AJAX + Websockets for client-server communication.
    \end{itemize}
    \item[]
    \headerrow {\textbf{Backend for an Online Coaching App}}{\underline{Bozobaka Labs Pvt. Ltd.}}
    \headerrow {\emph{Independent Internship}}{\emph{Jan-Aug 2018}}
    \begin{itemize}
    \setlength\itemsep{0.0em}
    \item Designed the entire backend system, focusing on reliability and speed, and successfully deployed its app.
    % App is currently being used by more than 5K users.
	\item Built the webservice using Loopback framework on NodeJS and aimed at keeping the APIs intuitive and extensible.
	\item MySQL+MongoDB were used for storage, prioritizing minimal duplication while optimizing time delay of frequent APIs.
	% \item Used MySQL + MongoDB databases and Redis for caching to make service faster and less compute intensive.
	\item Used Redis for caching frequently used and compute intensive APIs to make service faster and reduce server load.
    \item Created a dockerized OCR service using Tesseract for easing content creation and smarter student doubt resolution.
	% \item Employed MySQL and MongoDB databases to store structured and unstructured data.
    \end{itemize}
    % \item[]
    % \headerrow {\textbf{Debunking Neural Essay Scoring}}{Prof. Mausam, Mar-May 2018}
    % \begin{itemize}
    % \setlength\itemsep{0.0em}
	% \item Compared performance of Neural models vs Non-Neural Models on Automated Essay Scoring.
	% \item Designed various experiments to analyze and contrast their performance in sub-tasks of essay scoring.
	% \item Proposed various methods to mitigate the shortcomings of neural models on these important subtasks.
    % \end{itemize}
    \item[]
    \headerrow {\textbf{Automated Video Description}}{\underline{Prof. Subhashis Banerjee}}
    \headerrow {\emph{Summer Undergraduate Research Award}}{\emph{May-July 2017}}
    \begin{itemize}
    \setlength\itemsep{0.0em}
        \item Built an end to end model using deep learning to generate natural language summary of short video clips.
        \item Utilized transfer learning in encoder by finetuning state of art CNN (Inception V4) to encode individual video frames.
        \item Designed encoder decoder network architecture consisting of Multilayered LSTMs to achieve this translation.
        \item Experimented with Data Augmentation, Audio Features, Attention models, Loss metrics to improve performance.
    \end{itemize}
    % \item[]
    % \headerrow {\textbf{Real Time Video Augmentation}}{Prof. Subhashis Banerjee}
    % \headerrow {\emph{Computer Vision Project}}{\emph{Oct-Nov 2017}}
    % \begin{itemize}
    % \setlength\itemsep{0.0em}
	% \item Transported a human from a live green screen video stream into a custom 3D environment.
	% \item Used techniques like Chroma Keying in YCrCb space to get a matte from the green screen video.
	% \item Unreal Engine was used to create 3D environment and Light Mask was used for realistic keying output.
	% \item Unreal Engine was used to generate videos of a custom 3D environment from a moving camera.
	% \item Light mask and blending techniques were used for the keying output to get realistic results.
    % \end{itemize}
\end{list}
% \end{spacing}

\begin{absolutelynopagebreak}
\begin{longversion}
\section{Other Projects}
\begin{list} {\labelitemi}{\leftmargin=0em}
\setlength{\leftmargin}{0pt}
% \setlength\itemsep{5em}
%\begin{itemize}

% \headerrow {\textbf{\href{http://cap.rdv-iitd.com}{Rendezvous CAP Portal}}} {Independent Project, June - July 2017}
% \begin{itemize} \item[]
% Designed and developed the entire backend of Campus representative portal in NodeJS. Used Amazon DynamoDB for database and designed ingenious table structure to minimize the number of tables required. Implemented token based authentication along with OAuth Login and designed RESTful APIs for the entire competitive portal.
% \end{itemize}
\item[]
\headerrow {\textbf{Semantic Image Segmentation using Priors}}{Prof. Parag Singla}
\headerrow {\emph{Bachelor's Thesis Project}}{\emph{July 2018 - Present}}
\begin{itemize} \item[] 
Proposed a novel technique to improve multi-task learning inference using prior knowledge during test time. Tested this hypothesis on the task of semantic image segmentation using DeeplabV2 model on Pascal VOC dataset by forking a classification module to allow for multi-task learning. Currently doing rigorous tests on various architectures.
\end{itemize}

\item[]
\headerrow {\textbf{Real Time Video Augmentation}}{Prof. Subhashis Banerjee}
\headerrow {\emph{Computer Vision Project}}{\emph{Oct-Nov 2017}}
\begin{itemize} \item[]
    Transported a human from a live green screen video stream into a custom 3D environment. Used techniques like Chroma Keying in YCrCb space to get a matte from the green screen video. Unreal Engine was used to create 3D environment and light mask was used for realistic keying output. Demo video can be seen \href{https://github.com/ozym4nd145/Vision_Project/blob/master/demo_video/shashank_result.avi?raw=true}{here}.
\end{itemize}

\item[]
\headerrow {\textbf{Debunking Neural Essay Scoring}}{Prof. Mausam}
\headerrow {\emph{Natural Language Processing Project}}{\emph{Mar-May 2018}}
\begin{itemize} \item[] 
Analyzed various Neural and Non Neural systems for automated essay scoring and designed various experiments to compare their qualitative performance. Using the insights gained from these experiments, proposed methods to improve the qualitative performance of neural models. The link to the report is \href{https://www.cse.iitd.ac.in/~cs1150262/nlp.pdf}{here}.
\end{itemize}

% \item[]
% \headerrow {\textbf{Automated Theorem Prover}}{Prof. S. Arun Kumar, Nov-Dec 2017}
% \begin{itemize} \item[]
% Devised and Implemented a theorem prover based on Analytical Tableaux in SML. Proved invalidity of a First-order logic  formula  by  successively  applying  tableaux  rules  thus  finding  contradictions  and  closing  branches. Tested the prover with several tautologies.
% \end{itemize}

% \item[]
% \headerrow {\textbf{Video Stabilization}}{Prof. Subhashis Banerjee}
% \begin{itemize}\item[]
% Developed a tool to stabilize shaky video. It used Inverse Compositional Algorithm given in ``Lucas Kanade 20 Years On: A Unifying Framework" by Baker et al. Image warp function was calculated using the algorithm and the video frames were warped using it to stabilize the object in the video.
% \end{itemize}

% \item[]
% \headerrow {\textbf{Named Entity Recognition on Medical Dataset}}{Prof. Mausam, Mar-April 2018}
% \begin{itemize}\item[]
% Built an NER system for tagging diseases and treatments names. Experimented with a various of sequence taggers like LSTM, CRF and used features like POS tagging, semantic labelling, custom trained word embeddings, word shapes etc. Achieved best performance using listed features and BiLSTM with CRF.

% \end{itemize}

% \item[]
% \headerrow {\textbf{User Level Threads}}{}
% \begin{itemize}\item[]
% Implemented User level threads in XV6 Operating System and created busy waiting and triggered lock structures. Also created priority scheduler for these user level threads and handled starvation using priority donation and priorty inversion.
% \end{itemize}

% \item[]
% \headerrow {\textbf{Parallel Multi-level Graph Partitioning}}{Prof. Subodh Kumar, Mar 2018}
% \begin{itemize} \item[] 
% Parallelized the multi-level graph partitioning algorithm using OpenMP by resorting to recursive bisection heuristic. Created an efficient and scalable implementation of the algorithm involving coarsening, bisection and uncoarsening steps suggested in the paper "Parallel Multilevel Graph Partitioning" by Karypis et al, using parallel constructs.
% \end{itemize}

% \item[]
% \headerrow {\textbf{Multilingual OCR Service}}{Independent Project, Oct - Nov 2017}
% \begin{itemize} \item[] 
% Created a dockerized service for OCR of multilingual PDFs and Images. Used NodeJS and ExpressJS to create the webserver, Tesseract OCR for character recognition, Amazon S3 for remote files access and Amazon SES to mail notification. Whole service was built as a docker container to run on a multi-node cluster. Link to code is \href{https://github.com/ozym4nd145/ocr_web}{here}
% \end{itemize}


% \item[]
% \headerrow {\textbf{Auto Reimbursement Bot}}{Tower Hackathon Runner Up, June 2017}
% \begin{itemize} \item[] 
% Designed a dockerized service for automating bill reimbursements. Created a Slack bot which automatically extracted fields like amount, invoice no. from bill photos. Written in python, it used ImageMagick to process images, then Tesseract OCR for character recognition and finally a fuzzy parser with custom rules to find relevant fields.
% \end{itemize}

% \item[]
% \headerrow {\textbf{Deployment System}}{Dev Club, July - Aug 2018}
% \begin{itemize} \item[]
% Developed an end to end Deployment system using docker-compose and docker-machine. It used Github hooks to auto build docker images and Slack hooks to deploy these images on multiple nodes. It was written in Golang and used Ansible for multi-node deployment.
% \end{itemize}

% \item[]
% \headerrow {\textbf{Flashsubs}} {Independent Project, Mar - April 2016}
% \begin{itemize} \item[]
% Created a software for management of media library (Movies/TV Series) which intelligently fetches subtitles and renames obfuscated media files. It uses file hash to identify media in online database and uses filename for identification as last resort. It automatically downloads metadata like IMDb rating, plot, actors list etc. The link to project is \href{https://github.com/ozym4nd145/FlashSubs}{here}.
% \end{itemize}

% \item[]
% \headerrow {\textbf{Multicycle ARM Processor}}{Prof. Anshul Kumar, Mar - April 2017}
% \begin{itemize} \item[]
% Developed a Multicycle ARM processor in VHDL on FPGA. Modelled memory as slave and used AHB Lite bus for connection. Implemented UART for memory and extended processor with 7 segment display and interrupt controller.
% \end{itemize}

\item[]
\headerrow{ \textbf{Krivine and SECD Machine}} {Prof. Sanjiva Prasad}
\headerrow {\emph{Programming Language Project}}{\emph{April - May 2017}}
\begin{itemize} \item[] 
Implemented a compiler with Krivine and SECD machine in OCaml. A Lex Scanner converted program to tokens which were converted to an Abstract Syntax Tree using Recursive Descent Parser. The AST was type checked and a low level code was generated, which was executed by the machines. Machines also supported features like scoping,recursion etc.
\end{itemize}

\end{list}
%\end{itemize}

% \pagebreak

% \textit{*All projects are available in my github profile}
\end{longversion}

\begin{longversion}
\section{Relevant Courses}
\begin{itemize}
\setlength\itemsep{-1em}
\item \textbf{Computer Science:} \hfill \\
Natural Language Processing, Computer Vision, Artificial Intelligence, Machine Learning, Operating Systems, Computer Networks, Parallel Computing, Theory of Computation, Algorithm Design, Logic for CS, Programming Languages, Computer Architecture, Design Practices, Data Structures \& Algorithms, Discrete Mathematics, Digital Logic\\
% NLP, Computer Vision, AI, Machine Learning, Operating Systems, Computer Networks, Parallel Computing, Theory of Computation, Algorithm Design, Programming Languages, Computer Architecture, Design Practices.\\
\item \textbf{Mathematics and Electrical:}
% Linear Optimization, Signals \& Systems, Prob. \& Stochastic Processes, Calculus, Linear Algebra.\\
Linear Optimization, Signals \& Systems, Prob. \& Stochastic Processes, Calculus.
\end{itemize}
% \textit{*Courses currently pursuing}
\end{longversion}

\begin{longversion}
\section{Technical Skills}\begin{itemize}
\item \textbf{Programming Languages:}  C, \CPP, Python, Java, JavaScript, Golang, OCaml, Prolog, SML,VHDL, C\#, Matlab.
\item \textbf{Frameworks:} Docker, NodeJS, OpenCV, OpenMP, MPI, Git, Loopback, Django, MongoDB, Tensorflow, PyTorch.

\end{itemize}

\end{longversion}

\section{Extra Curricular Activities}
%\setlength{\leftmargin}{0pt}


\begin{itemize}
    \setlength\itemsep{0em}
    \item Co-founded \textbf{\href{http://devclub.iitd.ac.in/}{Software Development Club}} in IIT Delhi to foster development culture and created interesting projects.
    \item System Administrator in \textbf{Updaters Group}, handling the entire system architecture of CSE Dept. IIT Delhi.
    \item Overall Coordinator at \textbf{Coding Club}, responsible for organizing all competitive coding related events at IIT Delhi.
    \item \textbf{Technical Coordinator} at Tryst 2018, created and managed the entire back-end of the technical festival.
     
\end{itemize}
\end{absolutelynopagebreak}


\end{document}
